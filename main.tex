\documentclass[oneside,12pt]{book}
\textwidth=16cm \oddsidemargin=0pt \evensidemargin=0pt
\usepackage[english,russian]{babel}
\usepackage{amsmath,amsthm,amssymb}
\let\amslrcorner\lrcorner
\usepackage[T2A]{fontenc}
\usepackage[utf8]{inputenc}
\usepackage{amscd}
\usepackage{oldgerm}
\usepackage[matrix,arrow,curve]{xy}
\usepackage{tikz-cd}
\usepackage{comment}
%\usepackage{xcolor}
\usepackage{hyperref}
\usepackage{mathtools}
\usepackage{extpfeil}
\usepackage{cmap}
\usepackage{quiver}
\usepackage{MnSymbol}
\usepackage{mathrsfs}
\usetikzlibrary{babel}
\definecolor{linkcolor}{HTML}{799B03} % цвет ссылок
\definecolor{urlcolor}{HTML}{799B03} % цвет гиперссылок
\hypersetup{pdfstartview=FitH,  linkcolor=linkcolor,urlcolor=urlcolor, colorlinks=true}
\vfuzz2pt % Don't report over-full v-boxes if over-edge is small
\hfuzz2pt % Don't report over-full h-boxes if over-edge is small
% THEOREMS -------------------------------------------------------
\newtheorem{thm}{Теорема}[section]
\newtheorem{exercise}[thm]{Упражнение}
\newtheorem*{thm1}{Теорема}
\newtheorem{cor}[thm]{Следствие}
\newtheorem{lem}[thm]{Лемма}
\newtheorem{zad}[thm]{Задача}
\newtheorem{prop}[thm]{Предложение}
\newtheorem{prop*}[thm]{*Предложение}
\theoremstyle{definition}
\newtheorem{dfn}[thm]{Определение}
%\newtheorem{dfn}{Определение}[section]
\newtheorem{dfn*}{*Определение}[section]
\newtheorem*{defn1}{Определение}
\theoremstyle{remark}
\newtheorem{rem}[thm]{Замечание}
\newtheorem{example}{Пример}[section]
\numberwithin{equation}{section}
\renewcommand{\proofname}{Доказательство}
\numberwithin{equation}{section}

% MATH -----------------------------------------------------------
\newcommand{\set}[1]{\left\{#1\right\}}
\newcommand{\gen}[1]{\left\langle#1\right\rangle}
\newcommand{\dpr}[2]{\left(#1\shortmid#2\right)}% Dot product
\newcommand{\eps}{\varepsilon}
\newcommand{\rank}{\mathrm{rank}}
\newcommand{\coker}{\mathrm{coker}}
\newcommand{\Hom}{\mathrm{Hom}}
\newcommand{\codim}{\mathrm{codim}}
\newcommand{\To}{\longrightarrow}
\newcommand{\0}{\varnothing}
\newcommand{\N}{\mathbb{N}}
\newcommand{\Z}{\mathbb{Z}}
\newcommand{\Q}{\mathbb{Q}}
\newcommand{\CN}{\mathbb{C}}
\newcommand{\HN}{\mathbb{H}}
\newcommand{\End}{\mathrm{End}}
\newcommand{\M}{\mathrm{M}}
\newcommand{\R}{\mathbb{R}}
\newcommand{\RN}{\mathbb{R}}
\newcommand{\tr}{\mathrm{t}}
\newcommand{\trace}{\mathrm{tr}}
\newcommand{\Aut}{\mathrm{Aut}}
\newcommand{\spec}{\mathrm{Spec}}
\newcommand{\diag}{\mathrm{diag}}
\newcommand{\GL}{\mathrm{GL}}
\newcommand{\SU}{\mathrm{SU}}
\newcommand{\OG}{\mathrm{O}}
\newcommand{\SO}{\mathrm{SO}}
\newcommand{\m}{\mathrm{M}}
\newcommand{\un}{\mathrm{U}}
\newcommand{\UG}{\mathrm{U}}
\newcommand{\re}{\operatorname{Re}}
%\DeclareMathOperator{\re}{re}
\newcommand{\im}{\operatorname{Im}}
\newcommand{\ovl}{\overline}
\newcommand{\norm}[1]{\left\lVert#1\right\rVert}
\newcommand{\mf}[1]{{\mathfrak{#1}}}
\newcommand{\id}{\operatorname{id}}
\newcommand{\sm}{\setminus\set{0}}
\newcommand{\PS}{\operatorname{P}}
\DeclareMathOperator{\noreq}{\trianglelefteq}
\newcommand{\Co}{\mathbb{C}}
\begin{document}
	% !TeX encoding = windows-1251
\begin{titlepage}
	\center{������������ ����������� � ����� ���������� ��������� \\
        ����������� ��������������� ���������� ��������������� ���������� ������� ����������� \\
        ��������� ��������������� ������������ ����������������� �����������}

\vspace{1em}

	\center{��������-�������������� ��������� \\ ������� ��������������� ����������}

\vspace{5em}

	\center \textbf{�������� ������ \\ �� ����:}
	\center \textbf{\large{�������������� ������}}
	\center \textbf{�� ���������� "������������ ���������������� ���������"}
	\center {����������� 01.03.01 ����������}

\vspace{6em}

    \begin{flushright}
        \begin{tabular}{c c}
            �������� ������� ������ ��/�~���-2021~�� & ������� �.�. \\
            \multirow{3}{23em}{������� ������������ �������� ������-�������������� ����, ������~������� ��������������� ����������} & �������� �.�.
            \\
        \end{tabular}
    \end{flushright}

\vspace{\fill}

	\center{����� 2023}
\end{titlepage}
	\tableofcontents
	\newpage
	\chapter{�������� 1-� �����}

\section{�������� ����������}

\begin{dfn}
	��������� $\mathbb{N} = \mathbb{Z}_{>0}$.
\end{dfn}

����� $n \in \mathbb{N}$, ���� �� ������������� ���������, ��
�������� ��� ������������ $\mathbb{R}^n$ ����������� ����������, �
�� ������������ �������������� ����������.
��������� ����� $S^n$
\begin{equation}\label{sn}
	S^n = \set{(x_1,\ldots,x_{n+1}) \in \mathbb{R}^{n+1} : x_1^2+\ldots+x_{n+1}^2 = 1}
\end{equation}
� �������� ��� $B^n_r (a)$ ������� $r$ � ������� � ����� $a$
������������ $\mathbb{R}^n$
$$
B^n_r (a) = \set{(x_1,\ldots,x_n) \in \mathbb{R}^n : \sum_{i = 1}^n (x_i - a_i)^2 < r^2}.
$$
��� ������ ������������� ����� $x$ ���������
$$
\sigma(x) = \begin{cases}
	1, \text{ ���� } x > 0, \\
	0, \text{ ���� } x = 0, \\
	-1, \text{ ���� } x < 0.
\end{cases}
$$
��� $\epsilon= \set{\pm 1}$ ���������
$$
\mathbb{R}_\epsilon= \set{x \in \mathbb{R} : \sigma(x) = \epsilon},
$$
�������� ��������, ��� ���������� ��������� �����
�������� ��������� �������������� $\mathbb{R}$.



\begin{dfn}\label{1}
	��� $k\in \mathbb{N}$, $k\leq n+1$, $\epsilon = \set{\pm 1}$
	���������
	$$
	U_{\epsilon k} = \set{(x_1,\ldots,x_{n+1})\in S^n : \sigma(x_k) = \epsilon}.
	$$
\end{dfn}

\begin{lem}
	��������� $U_{\epsilon k}$ �� \ref{1} �������� ��������
	�������� $S^n$.
\end{lem}

\begin{proof}
	�������, ��� ������ ����� $x \in S_n$ �������� ���� �� �
	���� �� �������� $U_{\epsilon k}$. ����� ��������� ���������
	����� $x$ ���������, ������� ���� ��
	���� �� � ��������� ���������, �.�. ��� ����������
	$k \in \overline{1,n+1}$ ������� $x_k \neq 0$.
	����� ��������, ���� $x_k < 0$, �� $x$ �������� � $U_{-k}$,
	���� $x_k > 0$, �� $x$ �������� � $U_{k}$, ����� �������
	$$
	x \in U_{\sigma(\epsilon) k}, \; \forall x \in S^n.
	$$
	����� �������, �� ��������, ��� ��������� $U_{\epsilon k}$
	�������� �������� $S^{n}$.
	������ �������, ��� ��� $U_{\epsilon k}$ ������� (� $S^n$).
	���������� ��� ����� ������������ $\mathbb{R}^{n+1}$ � �������
	$k$-�� ���������� ����� ���� $\epsilon$:
	$$
	X_{\epsilon k} = \set{(x_1,\ldots,x_{n+1}) \in \mathbb{R}^{n+1}: \sigma(x_k) = \epsilon},
	$$
	�� ���� ��� ����������������.
	��������� $\pi_k$ --- ����������� �����������
	�������������� $\pi_k : \mathbb{R}^{n+1} \to \mathbb{R}$,
	������� ������ ������ ����� ������������ � $k$-��
	����������
	$$
	\pi_k (x_1,\ldots,x_{n+1}) = x_k.
	$$
	��������, ��� $\pi_k$ --- ����������� �����������,
	��� ��� ��� �������, �������� � ���� --- �����������.

	�������, ���
	$$
	X_{\epsilon k} = \pi_k^{-1} (\mathbb{R}_\epsilon).
	$$
	��� ��������, �������� ��������� ������������
	������������ ������������ ����������� --- ������,
	������������� $X_{\epsilon k}$ --- �������� ������������
	$\mathbb{R}^n$. ��������� �� $S^n$ ������������ �� $\mathbb{R}^{n+1}$,
	�������
	$$
	U_{\epsilon k} = S^n \cap X_{\epsilon k },
	$$
	--- �������� ������������ $S^n$.
\end{proof}


\begin{lem}\label{11}
	�����
	$$
	\epsilon = \set{\pm 1}, \;\; i \in \mathbb{N}, \;\; i\leq n+1, \;\; x\in U_{\epsilon i},
	$$
	�����
	$$
	\sum_{j\neq i}x_j^2 < 1.
	$$
\end{lem}

\begin{dfn}\label{2}
	������ ������������ �����������:
	$$
	\phi_{\epsilon k} : U_{\epsilon k} \to B_1^n (0),
	$$
	$$
	(x_1, \ldots, x_{n+1}) \mapsto (x_{1}, \ldots, x_{k-1}, x_{k+1}, \ldots, x_{n+1}),
	$$
	� ���� ����� \ref{11} ������ ����������� ���������.
\end{dfn}

\begin{dfn}\label{3}
	������ �����������:
	$$
	\psi_{\epsilon k} :  B_1^n (0) \to U_{\epsilon k}
	$$
	$$
	(x_1, \ldots, x_n) \mapsto (x_1, \ldots, x_{k-1}, \epsilon\sqrt{1 - \sum_{i =1}^n x_i^2}, x_k, \ldots, x_n).
	$$
\end{dfn}

\begin{thm}\label{LEMN1}
	���� $(U_{\epsilon k}, \phi_{\epsilon k})$, ��� $U_{\epsilon k}$ �
	$\phi_{\epsilon k}$ �������������� �� \ref{1} � \ref{2}
	�������� ������� ����� ����� $S^{n}$.
\end{thm}

\begin{proof}
	�������, ��� ����������� $\phi_{\epsilon k}$,  $\psi_{\epsilon k}$
	�� \ref{3} ������� �������. ����� $x \in B_{1}^{n}(0)$,
	�����
	$$
	\phi_{\epsilon k}\psi_{\epsilon k}(x) = \phi_{\epsilon k}\psi_{\epsilon k}(x_1, \ldots, x_n) =
	$$
	$$
	\phi_{\epsilon k}(x_1, \ldots, x_{k-1}, \epsilon\sqrt{1 - \sum_{i = 1}^n x_i^2}, x_{k}, \ldots, x_n) =
	$$
	$$
	(x_1,\ldots,x_n) = x.
	$$
	����� $x \in U_{\epsilon k}$, ����� ����� ��������� ���������
	$x$ ����� �������, �������������
	$$
	1 - \sum_{i \neq k}x_i^2 = x_k^2,
	$$
	������
	$$
	\sqrt{1 - \sum_{i \neq k}x_i^2} = |x_k|,
	$$
	� $\epsilon |x_k| = x_k$. �����,
	$$
	\psi_{\epsilon k}\phi_{\epsilon k}(x) = \psi_{\epsilon k}\phi_{\epsilon k}(x_1, \ldots, x_{n+1})=
	$$
	$$
	\psi_{\epsilon k}(x_{1}, \ldots, x_{k-1}, x_{k+1}, \ldots, x_{n+1}) =
	$$
	$$
	(x_{1}, \ldots, x_{k-1}, \epsilon\sqrt{1 - \sum_{i \neq k}x_i^2}, x_{k+1}, \ldots, x_{n+1}) =
	$$
	$$
	(x_{1}, \ldots, x_{k-1}, x_k, x_{k+1}, \ldots, x_{n+1}) = (x_1, \ldots, x_{n+1}) = x.
	$$
	����� �������, ��� ����������� $\phi_{\epsilon k}$ �
	$\psi_{\epsilon k}$ ����������, ��������� ��� �������
	�������� � ��� ����������� ����������. ����� �������, ��
	��������, ��� $\phi_{\epsilon k}$ --- ������������.

	����� $k_1, k_2 \in \set{1, \dots, n+1}$, ������
	$k_1 \neq k_2$, $\epsilon_1, \epsilon_2 \in \set{\pm 1}$.
	�����, ����������� $k_1 < k_2$ � ���������
	$$
	U_{\epsilon_1 k_1, \epsilon_2 k_2} :=
	U_{\epsilon_1 k_1} \cap U_{\epsilon_2 k_2},
	\;\;\;
	B := B_1^n(0),
	\;\;\;
	B_{\epsilon k} := \set{(x_1,\ldots,x_n) \in B \mid \sigma(x_k) = \epsilon}
	$$
	�����
	$$
	U_{\epsilon_1 k_1, \epsilon_2 k_2} =
	U_{\epsilon_1 k_1} \cap U_{\epsilon_2 k_2} =
	$$
	$$
	\set{ x \in S^n : \sigma(x_{k_1}) = \epsilon_1} \cap
	\set{ x \in S^n : \sigma(x_{k_2}) = \epsilon_2} =
	$$
	$$
	\set{x \in S^n : \sigma(x_{k_{1}}) =
		\epsilon_1, \sigma(x_{k_{2}}) = \epsilon_2} =
	$$
	$$
	\set{x \in U_{\epsilon_1 k_1} : \sigma(x_{k_{2}}) = \epsilon_2}
	=
	\set{x \in U_{\epsilon_2 k_2} : \sigma(x_{k_{1}}) = \epsilon_1}.
	$$
	�����,
	$$
	\phi_{\epsilon_1 k_1}(U_{\epsilon_1 k_1} \cap U_{\epsilon_2 k_2}) =
	\phi_{\epsilon_1 k_1}(\set{x \in U_{\epsilon_1 k_1} :
		\sigma(x_{k_{2}}) = \epsilon_2}) =
	$$
	$$
	\set{(x_1, \ldots, x_{k_1-1}, x_{k_1+1}, \ldots, x_{n+1}) \in \phi_{\epsilon_1 k_1}(U_{\epsilon_1 k_1}) = B_1^n (0) :
		\sigma(x_{k_{2}}) = \epsilon_2} =
	$$
	$$
	\set{(y_1, \ldots, y_{k_1-1},y_{k_1}, y_{k_1+1}, \ldots, y_n) \in B \mid \sigma(y_{k_2 - 1}) = \epsilon_2} =
	B_{\epsilon_2(k_2-1)}.
	$$

	$$
	\phi_{\epsilon_2 k_2}(U_{\epsilon_1 k_1} \cap U_{\epsilon_2 k_2}) =
	\phi_{\epsilon_2 k_2}(\set{(x_1,\ldots,x_{n+1})\in U_{\epsilon_2 k_2} :
		\sigma(x_{k_{1}}) = \epsilon_1}) =
	$$
	$$
	\set{(x_1, \ldots, x_{k_2-1}, x_{k_2+1}, \ldots, x_{n+1}) \in \phi_{\epsilon_2 k_2}(U_{\epsilon_2 k_2}) = B_1^n (0) : \sigma(x_{k_1}) = \epsilon_1} =
	$$
	$$
	\set{(x_1, \ldots, x_{k_2-1}, x_{k_2+1}, \ldots, x_{n+1}) \in B \mid \sigma(x_{k_1}) = \epsilon_1} =
	$$
	$$
	\set{y \in B \mid \sigma(y_{k_1}) = \epsilon_1} =
	B_{\epsilon_1 k_1}.
	$$
	����� �������, �� ��������
	$$
	\phi_{\epsilon_1 k_1}(U_{\epsilon_1 k_1} \cap U_{\epsilon_2 k_2}) =
	B_{\epsilon_2(k_2-1)},
	\;\;\;\;\;\;\;
	\phi_{\epsilon_2 k_2}(U_{\epsilon_1 k_1} \cap U_{\epsilon_2 k_2}) = B_{\epsilon_1 k_1}.
	$$
	��������� �����������:
	$$
	\tau_{\epsilon_2 k_2, \epsilon_1 k_1} := \left(\phi_{\epsilon_2 k_2}\Big|_{B_{\epsilon_1 k_1}, U_{\epsilon_1 k_1, \epsilon_2 k_2}}\right)\left(\phi_{\epsilon_1 k_1}\Big|_{B_{\epsilon_2(k_2 -1)}, U_{\epsilon_1 k_1, \epsilon_2 k_2}}\right)^{-1} :
	$$
	$$
	\phi_{\epsilon_1 k_1}(U_{\epsilon_1 k_1} \cap U_{\epsilon_2 k_2}) \to \phi_{\epsilon_2 k_2}(U_{\epsilon_1 k_1} \cap U_{\epsilon_2 k_2}).
	$$
	��������,
	$$
	\tau_{\epsilon_2 k_2, \epsilon_1 k_1}: B_{\epsilon_2(k_2 -1)} \to B_{\epsilon_1 k_1}.
	$$
	��� ���� $x \in B_{\epsilon_2(k_2 -1)}$ �����, ���
	$$
	\tau_{\epsilon_2 k_2, \epsilon_1 k_1}(x) = \phi_{\epsilon_2 k_2}(\phi_{\epsilon_1 k_1})^{-1}(x) =
	\phi_{\epsilon_2 k_2}\psi_{\epsilon_1 k_1}(x) = \phi_{\epsilon_2 k_2}\psi_{\epsilon_1 k_1}(x_{1}, \ldots, x_{n}) =
	$$
	$$
	\phi_{\epsilon_2 k_2}(x_1, \ldots, x_{k_{1}-1}, \epsilon_{1}\sqrt{1 - \sum_{i =1}^{n}x_i^2}, x_{k_{1}}, \ldots, x_n) =
	$$
	$$
	(x_1, \ldots, x_{k_{1}-1}, \epsilon_{1}\sqrt{1 - \sum_{i =1}^{n}x_i^2}, x_{k_{1}}, \ldots, x_{k_{2}}, x_{k_{2}+2}, \ldots, x_n).
	$$
	� ������ ����� $x \in B_{\epsilon_1 k_1}$, �����
	$$
	\tau_{\epsilon_1 k_1, \epsilon_2 k_2}(x) = \phi_{\epsilon_1 k_1}(\phi_{\epsilon_2 k_2})^{-1}(x) =
	\phi_{\epsilon_1 k_1}\psi_{\epsilon_2 k_2}(x) = \phi_{\epsilon_1 k_1}\psi_{\epsilon_2 k_2}(x_{1}, \ldots, x_{n}) =
	$$
	$$
	\phi_{\epsilon_1 k_1}(x_1, \ldots, x_{k_{2}-1}, \epsilon_{2}\sqrt{1 - \sum_{i =1}^{n}x_i^2}, x_{k_{2}}, \ldots, x_n) =
	$$
	$$
	(x_1, \ldots, x_{k_{1}-1}, x_{k_{1}+1}, \ldots, x_{k_{2}-1}, \epsilon_{2}\sqrt{1 - \sum_{i =1}^{n}x_i^2}, x_{k_{2}}, \ldots, x_n).
	$$
	�������� �������� �� ����������� $\tau_{\epsilon_2 k_2, \epsilon_1 k_1}$ ����������:
	$$
	\tau_{\epsilon_1 k_1, \epsilon_2 k_2}\tau_{\epsilon_2 k_2, \epsilon_1 k_1}(x) :=
	\tau_{\epsilon_2 k_2, \epsilon_1 k_1}^{-1}\tau_{\epsilon_2 k_2, \epsilon_1 k_1}(x_1, \ldots, x_n) =
	$$
	$$
	\tau_{\epsilon_1 k_1, \epsilon_2 k_2}(x_1, \ldots, x_{k_{1}-1}, \epsilon_{1}\sqrt{1 - \sum_{i =1}^{n}x_i^2}, x_{k_{1}}, \ldots, x_{k_{2}}, x_{k_{2}+2}, \ldots, x_n) =
	$$
	$$
	(x_1, \ldots, x_{k_{1}-1}, x_{k_{1}}, \ldots, x_{k_{2}}, \epsilon_{2}\sqrt{1 - \sum_{i \neq k_{2}+1}x_i^2}, x_{k_{2}+2}, \ldots, x_n) =
	$$
	$$
	(x_1, \ldots, x_{k_{1}-1}, x_{k_{1}}, \ldots, x_{k_{2}}, x_{k_{2}+1}, x_{k_{2}+2}, \ldots, x_n) =
	$$
	$$
	(x_1,\ldots,x_n) = x.
	$$
	� � �������� �������:
	$$
	\tau_{\epsilon_2 k_2, \epsilon_1 k_1}\tau_{\epsilon_1 k_1, \epsilon_2 k_2}(x) :=
	\tau_{\epsilon_2 k_2, \epsilon_1 k_1}\tau_{\epsilon_2 k_2, \epsilon_1 k_1}^{-1}(x_1, \ldots, x_n) =
	$$
	$$
	\tau_{\epsilon_2 k_2, \epsilon_1 k_1}(x_1, \ldots, x_{k_{1}-1}, x_{k_{1}+1}, \ldots, x_{k_{2}-1}, \epsilon_{2}\sqrt{1 - \sum_{i =1}^{n}x_i^2}, x_{k_{2}}, \ldots, x_n) =
	$$
	$$
	(x_1, \ldots, x_{k_{1}-1}, \epsilon_{1}\sqrt{1 - \sum_{i \neq k_1}x_i^2}, x_{k_{1}+1}, \ldots, x_{k_{2}-1}, x_{k_{2}}, \ldots, x_n) =
	$$
	$$
	(x_1, \ldots, x_{k_{1}-1}, x_{k_{1}}, x_{k_{1}+1}, \ldots, x_{k_{2}-1}, x_{k_{2}}, \ldots, x_n) =
	$$
	$$
	(x_1, \ldots x_n) = x.
	$$
	�������, ��� $\tau_{\epsilon_2 k_2, \epsilon_1 k_1}$ ��������
	������� ������������,
	��� ��� ��� ������� �������� � �� �������.
	����� �������, ���� $(U_{\epsilon k}, \phi_{\epsilon k})$ ��������
	������� ����� $S^{n}$.
\end{proof}




	\begin{thebibliography}{99} %библиография
		\bibitem{Volochkov}
		Волочков А.А. Введение в теорию конечных групп./ / А.А.Волочков. --- Пермь, 2020.---233 стр. 58.
		\bibitem{Kolmogorov}
		Колмогоров А.Н., Фомин С.В. Элементы теории функции и функционального анализа.
		\bibitem{Milnor}
		Милнор Д., Уоллес А. Дифференциальная топология начальный курс.
		\bibitem{Hirsh}
		Хирш М. Дифференциальная топология.
		\bibitem{Jurgen}
		Jurgen J. Riemannian Geometry and Geometric.
		\bibitem{Madsen}
		Madsen. From Calculus to Cohomology
		De Rnam no cohomology and characteristic classes
		\bibitem{Lee}
		John M. Lee Introduction to Smooth Manifolds.
	\end{thebibliography}


\end{document}